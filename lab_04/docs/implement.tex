\chapter{Технологическая часть}

В данном разделе будут рассмотрены средства реализации, представлены листинги реализаций алгоритма шифрования RSA и результаты тестирования.

\section{Средства реализации}

В данной работе для реализации был выбран язык программирования $C$. Данный язык удовлетворяет поставленным критериям по средствам реализации.

\section{Реализация алгоритма}

В листинге \ref{lst:RSA} представлена реализация алгоритма шифрования RSA.

\begin{center}
	\captionsetup{justification=raggedright,singlelinecheck=off}
	\begin{lstlisting}[label=lst:RSA,caption=RSA]
lli gcde(lli a, lli b, lli *x, lli *y) {
	if (a == 0) {
		*x = 0;
		*y = 1;
		return b;
	}
	lli x1, y1;
	lli d = gcde(b % a, a, &x1, &y1);
	*x = y1 - (b / a) * x1;
	*y = x1;
	return d;
}

lli mod_exp(lli num, lli exp, lli mod) {
	lli product = 1;
	
	while (exp > 0) {
		if (exp & 0x01) product = (product * num) % mod;
		num = (num * num) % mod;
		exp >>= 1;
	}
	
	return product;
}

void rsa_key_gen(rsa_key_t *public_key, rsa_key_t *private_key) {
	lli N = 0, phi = 0;
	lli e = E;
	
	N = P * Q;
	phi = (P - 1) * (Q - 1);
	
	lli x, y;
	gcde(e, phi, &x, &y);
	while (x < 0) x = x + phi;
	
	public_key->mod = N;
	public_key->exp = e;
	
	private_key->mod = N;
	private_key->exp = x;
}

lli rsa(const lli data, const struct rsa_key_t key) {
	return mod_exp(data, key.exp, key.mod);
}
	\end{lstlisting}
\end{center}

\section{Тестирование}

Для тестирование написанной программы сначала генерировались ключи. Потом подписывался файл и проверялась его подпись.

\begin{table}[h]
	\begin{center}
		\begin{threeparttable}
			\captionsetup{justification=raggedright,singlelinecheck=off}
			\caption{\label{tbl:functional_RSA} Функциональные тесты ЭЦП}
			\begin{tabular}{|c|c|c|c|}
				\hline
				Входной файл & Ожидаемый результат & Результат \\ 
				\hline
				Текстовый файл & Успешная проверка &Успешная проверка\\
				\hline
				1 байтовый файл & Успешная проверка &Успешная проверка\\
				\hline
				Пустой файл & Успешная проверка &Успешная проверка\\
				\hline
				Архив & Успешная проверка &Успешная проверка\\
				\hline
				PNG изображение & Успешная проверка &Успешная проверка\\
				\hline
				2 разных файла & Проваленная проверка &Проваленная проверка\\
				\hline
			\end{tabular}
		\end{threeparttable}
	\end{center}
\end{table}
\FloatBarrier

Для тестирование MD5 был введён дополнительный режим работы программы, позволяющий получить хеш файла. Результат работы программы сравнивался с результатом md5sum с помощью команды cmp.

\begin{table}[h]
	\begin{center}
		\begin{threeparttable}
			\captionsetup{justification=raggedright,singlelinecheck=off}
			\caption{\label{tbl:functional_MD5} Функциональные тесты MD5}
			\begin{tabular}{|c|c|c|c|}
				\hline
				Входной файл & Ожидаемый результат & Результат \\ 
				\hline
				Текстовый файл & Пустой вывод &Пустой вывод\\
				\hline
				1 байтовый файл & Пустой вывод &Пустой вывод\\
				\hline
				Пустой файл & Пустой вывод &Пустой вывод\\
				\hline
				Архив & Пустой вывод &Пустой вывод\\
				\hline
				PNG изображение & Пустой вывод &Пустой вывод\\
				\hline
			\end{tabular}
		\end{threeparttable}
	\end{center}
\end{table}
\FloatBarrier
