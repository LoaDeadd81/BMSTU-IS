\chapter{Аналитическая часть}

\section{Алгоритм RSA}

Ассиметричный алгоритм криптографии RSA, датой возникновения концепции которого считается 1976 год сейчас очень активно используется для обмена данными, верификацией источника программного обеспечения и в других сферах, где необходимо обмениваться данными или верифицировать отправителя.

В отличии от симметричных алгоритмов шифрования, имеющих всего один ключ для шифрования и расшифровки информации, в алгоритме RSA используется 2 ключа --- открытый (публичный) и закрытый (приватный).

Публичный ключ шифрования передаётся по открытым каналам связи, а приватный всегда держится в секрете. В ассиметричной криптографии и алгоритме RSA, в частности, публичный и приватный ключи являются двумя частями одного целого и неразрывны друг с другом.

В основе работы данного алгоритма лежит математический объект, называемый перестановкой с потайным входом --- функция, которая преобразует число x в число y в том же диапазоне, так что вычислить y по x легко, зная открытый ключ, но вычислить x по y практически невозможно, если не знать закрытого ключа -- потайного входа.(Можете считать, что x – открытый текст, а y – шифртекст.)

Кроме шифрования, RSA используется для создания цифровых подписей, когда только владелец закрытого ключа может подписать сообщение, а наличие открытого ключа позволяет любому желающему проверить достоверность подписи. 

Из--за использования возведения в степень данный алгоритм кодирует ноль нулём, поэтому его в основном используют только для передачи ключа симметричного алгоритма. Также недостатком является скорость работы, из--за большого количества операций возведения в степень больших чисел. Стойкость алгоритма базируется на сложности решения задачи факторизации.
 
На рисунке \ref{img:rsa} представлен пример работы алгоритма RSA.

\imgScale{0.5}{rsa}{Пример работы RSA}
\FloatBarrier

Алгоритм RSA видит сообщение как большое число, а само шифрование заключается, по существу, в умножении больших чисел. RSA видит открытый текст как положительное целое число от $1$ до $n - 1$, где $n$ -- модуль. При перемножении таких чисел получается третье число, удовлетворяющее тем же условиям. Эти числа образуют мультипликативную группу целых чисел по модулю $n$. Для нахождения числа элементов группы, когда $n$ не является простым числом, используется функция Эйлера $\varphi(n)$. Она даёт количество чисел, меньших $n$ и взаимно простых с $n$, т.~е. как раз количество элементов группы. Если $n$ разложить в произведение простых чисел $n = p_1 * ... * p_m$, то $\varphi(n) = (p_1 - 1) * ... * (p_m - 1)$. RSA имеет дело только с числами $n$, являющимися произведением двух больших простых чисел, $n = p*q$, следовательно $\varphi(n) = (p - 1)*(q - 1)$.

Если задан модуль $n$ и число $e$, называемое открытым показателем степени, то перестановка с потайным входом преобразует число $x$ в $y = x^e mod n$. $n$ и $e$ составляют открытый ключ. Чтобы получить $x$ по $y$, нам нужно еще одно число, $d$, такое что: $x = y^d mod n = (x^e mod n)^d mod n = x^{ed} mod n = x$. Закрытый ключ состоит из $n$ и $d$.

Очевидно, что $d$ -- не любое число  а такое, что $e * d = 1$. Точнее, должно иметь место равенство $ed = 1 mod \varphi(n)$. Заметим, что вычисление производится по модулю $\varphi(n)$, а не по модулю $n$, потому что показатели степени ведут себя как индексы элементов, которых всего  $\varphi(n)$.

\subsection{Генерация ключей}

Процедура создания публичного и приватного ключей:

\begin{itemize}[label=---]
    \item выбираем два случайных простых числа $p$ и $q$;
    \item вычисляем их произведение: $n = p * q$;
    \item вычисляем функцию Эйлера: $\varphi(n) = (p-1) * (q-1)$;
    \item выбираем число $e$, которое меньше $\varphi(n)$ и является взаимно простым с $\varphi(n)$;
    \item ищем число $d$, обратное числу $e$ по модулю $\varphi(n)$, т.е. остаток от деления $(d*e)$ и $\varphi(n)$ должен быть равен 1. Найти его можно через расширенный алгоритм Евклида.
\end{itemize}

После произведённых вычислений получаем публичный ключ $\{e, n\}$ и приватный ключ $\{d, n\}$. Желательно брать $e$ так, чтоб в его двоичной записи было минимальное количество единиц, что позволить совершать меньше операций в алгоритме быстрого возведения в степень.

\section{Алгоритм хеширования MD5}

Хэш-функция предназначена для свертки входного массива любого размера в битовую строку, для MD5 длина выходной строки равна 128 битам. Если имеется два массива, и необходимо быстро сравнить их на равенство, то хэш-функция поможет сделать это, если у двух массивов хэши разные, то массивы гарантировано разные, а в случае равенства хэшей --- массивы скорее всего равны. Однако чаще всего хэш-функции используются для проверки уникальности пароля, файла, строки и тд.

Алгоритм включает в себя 5 основных действий:

\begin{itemize}[label=---]
    \item выравнивание потока. Сначала к концу потока дописывают единичный бит. Затем добавляют некоторое число нулевых бит такое, чтобы новая длина потока стала сравнима с 448 по модулю 512. Необходимо для следующего этапа;
    \item добавление длины сообщения. В конец сообщения дописывают 64-битное представление длины данных (количество бит в сообщении) до выравнивания. Сначала записывают младшие 4 байта, затем старшие. осле этого длина потока станет кратной 512. Вычисления будут основываться на представлении этого потока данных в виде массива слов по 512 бит;
    \item инициализация буфера. Для вычислений инициализируются четыре переменные размером по 32 бита, в этих переменных будут храниться результаты промежуточных вычислений.
    \item вычисление в цикле. Происходит перемешивание 16 раундов по 4 этапа;
    \item результат вычислений.
\end{itemize}

На рисунке \ref{img:MD5} представлен пример работы алгоритма хеширования MD5.

\imgScale{0.7}{MD5}{Пример работы MD5}
\clearpage

\section{Электронно цифровая подпись}

Система RSA может использоваться не только для шифрования, но и для цифровой подписи. Предположим, что Алисе нужно отправить Бобу сообщение $m$, подтверждённое электронной цифровой подписью. Схема обема сообщениями с ЭЦП представлена на рисунке \ref{img:sign}.

\imgScale{0.6}{sign}{Схема работы ЭЦП}
\FloatBarrier

Действия Алисы:
\begin{itemize}[label=---]
	\item взять открытый текст $m$;
	\item создать ЭЦП с помощью своего секретного ключа, получить хэш файла и зашифровать его;
	\item передать пару из файла и подписи.
\end{itemize}

Действия Боба:
\begin{itemize}[label=---]
	\item принять пару из файла и подписи;
	\item расшифровать подпись открытым колючем;
	\item получить хэш принятого файла;
	\item проверить равенство двух хешей.
\end{itemize}

Важное свойство цифровой подписи заключается в том, что её может проверить каждый, кто имеет доступ к открытому ключу её автора. Один из участников обмена сообщениями после проверки подлинности цифровой подписи может передать подписанное сообщение ещё кому-то, кто тоже в состоянии проверить эту подпись. 











