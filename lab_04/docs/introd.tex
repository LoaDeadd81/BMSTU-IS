\chapter*{Введение}
\addcontentsline{toc}{chapter}{Введение}

Цифровая подпись --- это криптографический механизм, который используется для проверки подлинности и целостности цифровых данных. Мы можем рассматривать его как цифровую версию обычных рукописных подписей, но с более высоким уровнем сложности и безопасности.

Выражаясь простыми словами, мы можем описать цифровую подпись как код прикрепленный к сообщению или документу. После его генерации он выступает в качестве доказательства того, что сообщение не было подделано на протяжении своего пути от отправителя к получателю.

\textbf{Цель} --- реализация программы создания и проверки электронной подписи для документа с использованием алгоритма RSA и алгоритмов хеширования MD5.

Для достижения поставленной цели необходимо выполнить следующие задачи:
\begin{itemize}[label=---]
    \item изучить алгоритм работы алгоритма RSA;
    \item изучить алгоритм хеширования MD5;
    \item реализовать в виде программы алогритм RSA и алгоритм MD5;
    \item обеспечить шифрование и расшифровку произвольного файла с использованием разработанной программы;
    \item предусмотреть работу программы с пустым, однобайтовым файлом и с файлами архива (rar, zip или др.).
\end{itemize}
