\chapter*{Введение}
\addcontentsline{toc}{chapter}{Введение}

Сжатие данных --- алгоритмическое преобразование данных, производимое с целью уменьшения занимаемого ими объёма. Применяется для более рационального использования устройств хранения и передачи данных. Синонимы — упаковка данных, компрессия, сжимающее кодирование, кодирование источника. Обратная процедура называется восстановлением данных (распаковкой, декомпрессией). Сжатие основано на устранении избыточности, содержащейся в исходных данных. Простейшим примером избыточности является повторение в тексте фрагментов (например, слов естественного или машинного языка). Подобная избыточность обычно устраняется заменой повторяющейся последовательности ссылкой на уже закодированный фрагмент с указанием его длины. Другой вид избыточности связан с тем, что некоторые значения в сжимаемых данных встречаются чаще других. Сокращение объёма данных достигается за счёт замены часто встречающихся данных короткими кодовыми словами, а редких — длинными (энтропийное кодирование). Сжатие данных, не обладающих свойством избыточности (например, случайный сигнал или белый шум, зашифрованные сообщения), принципиально невозможно без потерь. Сжатие без потерь позволяет полностью восстановить исходное сообщение, так как не уменьшает в нем количество информации, несмотря на уменьшение длины. Такая возможность возникает только если распределение вероятностей на множестве сообщений не равномерное, например часть теоретически возможных в прежней кодировке сообщений на практике не встречается.

\textbf{Цель} --- Разработка алгоритма сжатия информации LZW.

Для достижения поставленной цели необходимо выполнить следующие задачи:
\begin{itemize}[label=---]
    \item изучить алгоритм работы LZW;
    \item реализовать в виде программы алгоритм LZW;
    \item обеспечить сжатие и разжатие произвольного файла с использованием разработанной программы, рассчитывать коэффициент сжатия;
    \item предусмотреть работу программы с пустым, однобайтовым файлом и с файлами архива (rar, zip или др.).
\end{itemize}
