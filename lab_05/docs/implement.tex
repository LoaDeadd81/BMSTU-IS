\chapter{Технологическая часть}

В данном разделе будут рассмотрены средства реализации, представлены листинги реализаций алгоритма шифрования LZW и результаты тестирования.

\section{Средства реализации}

В данной работе для реализации был выбран язык программирования $C$. Данный язык удовлетворяет поставленным критериям по средствам реализации.

\section{Реализация алгоритма}

В листинге \ref{lst:RSA} представлена реализация алгоритма шифрования RSA.

\begin{center}
	\captionsetup{justification=raggedright,singlelinecheck=off}
	\begin{lstlisting}[label=lst:RSA,caption=RSA]
int encode(int in_fd, int out_fd) {
	auto dict = get_dict();
	uint16_t code = NEW_CODE;
	
	int rc = 0;
	char ch;
	if ((rc = read(in_fd, &ch, sizeof(char))) < 0) {
		perror("read char error");
		return -1;
	}
	if(rc == 0) return 0;
	string p(1, ch);
	
	while ((rc = read(in_fd, &ch, sizeof(char))) > 0) {
		if (dict.contains(p + ch)) p += ch;
		else {
			if (write(out_fd, &dict[p], sizeof(uint16_t)) < 0) {
				perror("write code error");
				return -1;
			}
			dict[p + ch] = code++;
			p = ch;
		}
		if(code == INT16_MAX){
			if (!p.empty()) {
				if (write(out_fd, &dict[p], sizeof(uint16_t)) < 0) {
					perror("write code error");
					return -1;
				}
			}
			
			dict = get_dict();
			code = NEW_CODE;
			
			if ((rc = read(in_fd, &ch, sizeof(char))) < 0) {
				perror("read char error");
				return -1;
			}
			if(rc == 0) return 0;
			p = ch;
			
			uint16_t tmp = CLR_CODE;
			if (write(out_fd, &tmp, sizeof(uint16_t)) < 0) {
				perror("write clr code error");
				return -1;
			}
		}
	}
	if (rc < 0) {
		perror("read char error");
		return -1;
	}
	if (!p.empty()) {
		if (write(out_fd, &dict[p], sizeof(uint16_t)) < 0) {
			perror("write code error");
			return -1;
		}
	}
	
	return 0;
}

int decode(int in_fd, int out_fd) {
	auto dict = get_inv_dict();
	uint16_t code = NEW_CODE;
	
	string s, entry;
	uint16_t k;
	int rc = 0;
	if ((rc = read(in_fd, &k, sizeof(uint16_t))) < 0){
		perror("read code error");
		return -1;
	}
	if(rc == 0) return 0;
	s = dict[k];
	if(write_str(out_fd, s) < 0){
		perror("write str error");
		return -1;
	}
	
	while ((rc = read(in_fd, &k, sizeof(uint16_t))) > 0) {
		if(k == CLR_CODE){
			dict = get_inv_dict();
			code = NEW_CODE;
			
			if ((rc = read(in_fd, &k, sizeof(uint16_t))) < 0){
				perror("read code error");
				return -1;
			}
			if(rc == 0) return 0;
			s = dict[k];
			if(write_str(out_fd, s) < 0){
				perror("write str error");
				return -1;
			}
			continue;
		}
		
		if (dict.contains(k))
		entry = dict[k];
		else if (k == code)
		entry = s + s[0];
		else
		throw exception();
		
		if(write_str(out_fd, entry) < 0){
			perror("write str error");
			return -1;
		}
		dict[code++] = s + entry[0];
		s = entry;
	}
	if (rc < 0) {
		perror("read code error");
		return -1;
	}
	
	return 0;
}

	\end{lstlisting}
\end{center}

\section{Тестирование}

Для тестирование написанной программы сжимались и разжимались файлы и проверялся коэффициент сжатия.

\begin{table}[h]
	\begin{center}
		\begin{threeparttable}
			\captionsetup{justification=raggedright,singlelinecheck=off}
			\caption{\label{tbl:functional_RSA} Функциональные тесты ЭЦП}
			\begin{tabular}{|c|c|c|c|}
				\hline
				Входной файл & Ожидаемый результат & Результат \\ 
				\hline
				Текстовый файл & Файлы совпали & Файлы совпали\\
				 & КС < 1 & КС = 0.5 \\
				\hline
				1 байтовый файл & Файлы совпали & Файлы совпали\\
				& КС < 1 & КС = 0.5 \\
				\hline
				Пустой файл & Файлы совпали & Файлы совпали\\
				& КС = -//- & КС = -//- \\
				\hline
				Архив & Файлы совпали & Файлы совпали\\
				& КС < 1 & КС = 0.56 \\
				\hline
				PNG изображение & Файлы совпали & Файлы совпали\\
				& КС < 1 & КС = 0.69 \\
				\hline
				Текст из множества & Файлы совпали & Файлы совпали\\
				символов <<a>> & КС > 1 & КС = 11 \\
				\hline
				Однотонная картинка & Файлы совпали & Файлы совпали\\
				& КС > 1 & КС = 194 \\
				\hline
			\end{tabular}
		\end{threeparttable}
	\end{center}
\end{table}
\FloatBarrier
