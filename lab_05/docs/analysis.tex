\chapter{Аналитическая часть}

\section{Алгоритм LZW}

Алгоритм Лемпеля -- Зива -- Уэлча (Lempel-Ziv-Welch, LZW) --- это универсальный алгоритм сжатия данных без потерь, созданный Авраамом Лемпелем, Яаковом Зивом и Терри Велчем. Он был опубликован Велчем в 1984 году в качестве улучшенной реализации алгоритма LZ78, опубликованного Лемпелем и Зивом в 1978 году. Алгоритм разработан так, чтобы его было достаточно просто реализовать как программно, так и аппаратно. Акроним <<LZW>> указывает на фамилии изобретателей алгоритма: Лемпель, Зив и Велч.

Опубликование алгоритма LZW произвело большое впечатление на всех специалистов по сжатию информации. За этим последовало большое количество программ и приложений с различными вариантами этого метода.

Этот метод позволяет достичь одну из наилучших степеней сжатия среди других существующих методов сжатия графических данных, при полном отсутствии потерь или искажений в исходных файлах. В настоящее время используется в файлах формата TIFF, PDF, GIF, PostScript и других, а также отчасти во многих популярных программах сжатия данных (ZIP, ARJ, LHA).

Процесс сжатия выглядит следующим образом: последовательно считываются символы входного потока и происходит проверка, существует ли в созданной таблице строк такая строка. Если такая строка существует, считывается следующий символ, а если строка не существует, в поток заносится код для предыдущей найденной строки, строка заносится в таблицу, а поиск начинается снова.

Алгоритм кодирования:
\begin{enumerate}
	\item Все возможные символы заносятся в словарь. Во входную фразу X заносится первый символ сообщения;
	\item Считать очередной символ Y из сообщения;
	\item Если Y --- это символ конца сообщения, то выдать код для X, иначе: если фраза XY уже имеется в словаре, то присвоить входной фразе значение XY и перейти к Шагу 2, иначе выдать код для входной фразы X, добавить XY в словарь и присвоить входной фразе значение Y и перейти к Шагу 2
\end{enumerate}

Программы кодирования и декодирования должны начинаться с одного и того же начального словаря. Для декодирования на вход подается только закодированный текст, поскольку алгоритм LZW может воссоздать соответствующую таблицу преобразования непосредственно по закодированному тексту. Декодер LZW сначала считывает индекс (целое число), ищет этот индекс в словаре и выводит подстроку, связанную с этим индексом. Первый символ этой подстроки конкатенируется с текущей рабочей строкой. Эта новая конкатенация добавляется в словарь (подобно тому, как подстроки были добавлены во время сжатия). Затем декодированная строка становится текущей рабочей строкой (текущий индекс, т.е. подстрока, запоминается), и процесс повторяется.

Алгоритм декодирования:
\begin{enumerate}
	\item Все возможные символы заносятся в словарь. Во входную фразу X заносится первый символ сообщения;
	\item Считать очередной код Y из сообщения;
	\item Если Y --- это конец сообщения, то выдать символ, соответствующий коду X, иначе: если фразы под кодом XY нет в словаре, вывести фразу, соответствующую коду X, а фразу с кодом XY
	занести в словарь, Иначе присвоить входной фразе код XY и перейти к Шагу 2
\end{enumerate}










