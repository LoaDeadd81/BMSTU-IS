\chapter{Технологическая часть}

В данном разделе будут рассмотрены средства реализации, а также представлены листинги реализаций алгоритма шифрования AES.

\section{Средства реализации}
В данной работе для реализации был выбран язык программирования $C$. Данный язык удоволетворяет поставленным критериям по средствам реализации.

\section{Реализация алгоритма}

В листингах \ref{lst:encode_block} и \ref{lst:decode_block} представлена реализация алгоритма шифрования AES с использованием режима CFB.

\begin{center}
    \captionsetup{justification=raggedright,singlelinecheck=off}
    \begin{lstlisting}[label=lst:encode_block,caption=Шифрование AES]
void encode_block(uint8_t *dst, uint8_t *src, uint32_t *keys) {
	uint8_t block[16];
	
	copy_t_block(block, src);
	
	add_round_key(block, keys);
	for (uint8_t round = 1; round < ROUND_NUM; ++round) {
		sub_bytes(block);
		shift_rows(block);
		mix_columns(block);
		add_round_key(block, keys + (NB * round));
	}
	sub_bytes(block);
	shift_rows(block);
	add_round_key(block, keys + (NB * ROUND_NUM));
	
	copy_t_block(dst, block);
}
\end{lstlisting}
\end{center}

\begin{center}
\captionsetup{justification=raggedright,singlelinecheck=off}
\begin{lstlisting}[label=lst:decode_block,caption=Дешифрование AES]
void decode_block(uint8_t *dst, uint8_t *src, uint32_t *keys) {
	uint8_t block[16];
	
	copy_t_block(block, src);
	
	add_round_key(block, keys + (NB * ROUND_NUM));
	for (int8_t round = ROUND_NUM - 1; round > 0; --round) {
		inv_shift_rows(block);
		inv_sub_bytes(block);
		add_round_key(block, keys + (NB * round));
		inv_mix_columns(block);
	}
	inv_shift_rows(block);
	inv_sub_bytes(block);
	add_round_key(block, keys);
	
	copy_t_block(dst, block);
}
\end{lstlisting}
\end{center}
\FloatBarrier

\section{Тестирование}

Для тестирование написанной программы файл шифровался и дешифровался и сравнивалось их содержимое.

\begin{table}[h]
	\begin{center}
		\begin{threeparttable}
			\captionsetup{justification=raggedright,singlelinecheck=off}
			\caption{\label{tbl:functional_} Функциональные тесты}
			\begin{tabular}{|c|c|c|c|}
				\hline
				Входной файл & Ожидаемый результат & Результат \\ 
				\hline
				1\_in.txt (9 bytes) & 1\_out.txt (9 bytes) & 1\_out.txt (9 bytes)\\
				<<encode\_block me>> & <<encode\_block me>> &<<encode\_block me>>\\
				\hline
				2\_in.txt (1 bytes) & 2\_out.txt (1 bytes) & 2\_out.txt (1 bytes)\\
				<<a>> & <<a>> &<<a>>\\
				\hline
				3\_in.txt (0 bytes) & 3\_out.txt (0 bytes) & 3\_out.txt (0 bytes)\\
				<<>> & <<>> &<<>>\\
				\hline
				4\_in.tar.gz (21 Kb) & 4\_out.tar.gz (21 Kb) & 4\_out.tar.gz (21 Kb)\\
				test.png & test.png & test.png\\
				\hline
			\end{tabular}
		\end{threeparttable}
	\end{center}
\end{table}
\FloatBarrier

\section*{Вывод}

Были представлен листинг реализации алгоритма работы энигмы. Также в данном разделе была приведена информация о выбранных средствах для разработки алгоритмов.
