\chapter*{Введение}
\addcontentsline{toc}{chapter}{Введение}

Шифрование информации --- занятие, которым человек занимался ещё до начала первого тысячелетия, занятие, позволяющее защитить информацию от посторонних лиц. 

Существует множество методов шифрования. Одним из них является блочное шифрование. Блочные шифры оперируют группами бит фиксированной длины --- блоками, характерный размер которых меняется в пределах 64--256 бит. Одним из самых распространённых шифров является AES. Это симметричный алгоритм блочного шифрования (размер блока 128 бит, ключ 128/192/256 бит), принятый в качестве стандарта шифрования правительством США по результатам конкурса AES. Этот алгоритм хорошо проанализирован и сейчас широко используется, как это было с его предшественником DES.

\textbf{Цель} --- реализация программы шифрования симметричным алгоритмом AES с применением режима шифрования CFB.

Для достижения поставленной цели необходимо выполнить следующие задачи:
\begin{itemize}[label=---]
	\item изучить алгоритм работы алгоритма AES;
	\item изучить режим шифрования CFB;
	\item реализовать в виде программы алогритм шифрования AES с применением режима шифрования CFB;
	\item обеспечить шифрование и расшифровку произвольного файла с использованием разработанной программы;
	\item предусмотреть работу программы с пустым, однобайтовым файлом и с файлами архива (rar, zip или др.).
\end{itemize}

