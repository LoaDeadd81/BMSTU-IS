\chapter{Аналитическая часть}

\section{Алгоритм AES}

AES --- симметричный итеративный блочный алгоритм шифрования, базирующийся на SP--сети. В отличии от сетей Фейстеля, в SP--сети шифруется весь текст. Особенностями алгоритма является: представление шифруемого блока в виде двумерного байтового массива, шифрование за один раунд всего блока данных (байт-ориентированная структура), выполнение криптографических преобразований, как над отдельными байтами массива, так и над его строками и столбцами. Последний пункт обеспечивает рассеивание.

AES позволяет использовать три различных ключа длиной 128,192 или 256 бит. В зависимости от длинны ключа варьируется количество раундов. Для 128 битного ключа используется 10 раундов. Перед началом раундов производится расширение ключа от одного 128 битного до одиннадцати 128 битных ключей. Дополнительный ключ используется для начальной подстановки, которая нужна для единообразия алгоритмов кодирования и декодирования. Общая схема алгоритма представлена на рисунке \ref{img:AES}.

Каждый раунд шифрования состоит из 4 этапов: SubBytes, ShiftRows, MixColumns, AddRoundKey. Последний раунд не включает операцию MixColumns, для избежания бесполезных вычислений, поскольку MixColumns линейная, ее воздействие в самом последнем раунде можно аннулировать, скомбинировав биты способом, не зависящим ни от их значения, ни от ключа.

Для описания алгоритма используется конечное поле Галуа $GF(2^8)$, построенное как расширение поля$GF(2)$ по модулю неприводимого многочлена $m(x) = x^8 + x^4 + x^3 + x + 1$, аналог простого числа. Элементами поля являются многочлены 7 степени. Коэффициентами являются биты. Сложение подставляет собой операцию XOR. Умножение байт выполняется с помощью представления их многочленами и перемножения по обычным алгебраическим правилам, полученное произведение необходимо привести по модулю многочлена $m(x)$. Для любого ненулевого битового многочлен в поле существует многочлен обратный к нему по умножению. Для нахождения обратного элемента используют расширенный алгоритм Эвклида.

\imgScale{0.8}{AES}{Схема AES}
\FloatBarrier

\textbf{SubBytes} заменяет каждый байт блока другим байтом согласно S--блоку, который представляет собой таблицу из 256 элементов. SubBytes привносит нелинейные операции, повышающие криптографическую стойкость. Операция выполняет нелинейную замену байтов, выполняемую независимо с каждым байтом матрицы. Замена обратима и
построена путем комбинации двух преобразований над входным байтом: нахождение обратного, умножение его на многочлен $x^4 + x^3 + x^2 + x + 1$ и суммирование с многочленом $x^6 + x^5 + x + 1$. Эту операцию можно представить в матричном виде. Нелинейность преобразования обусловлена нелинейностью инверсии, а обратимость -- обратимостью матрицы. В программной реализации используется массив с уже подсчитанными подстановками для каждого элемента поля. Схема на рисунке \ref{img:SB}.

\imgScale{0.8}{SB}{Схема SubBytes}.
\FloatBarrier

\textbf{ShiftRows} циклически сдвигает i-ю строку на i позиций, где i изменяется от 0 до 3. Без ShiftRows изменения в любом столбце не оказывали бы влияния на другие столбцы. Схема представлена на рисунке \ref{img:SR}.

\imgScale{0.8}{SR}{Схема ShiftRows}.
\FloatBarrier

В MixColumns используются многочлены третьей степени с коэффициентами из конечного поля $GF(2^8)$ и имеют вид $a(x) = a_3x^3 + a_2x^2 + a_1x + a_0$. Таким образом, в этих многочленах в роли коэффициентов при неизвестных задействованы байты вместо бит. Введём дополнительно многочлен $b(x) = b_3x^3 + b_2x^2 + b_1x + b_0$, тогда сложение определено как $a(x) + b(x) = (a_3 \oplus b_3)x^3 + (a_2 \oplus b_2)x^2 + (a_1 \oplus b_1)x + (a_0 \oplus b_0)$. Умножение аналогично умножению многочленов, взятое по модулю $x^4 + 1$. 

\textbf{MixColumns} применяет одно и то же линейное преобразование к каждому из четырех столбцов состояния. Без MixColumns изменение одного байта не влияло бы на остальные байты состояния. Каждый столбец этой матрицы принимается за многочлен над полем $GF(2^8)$ и умножается на фиксированный многочлен $c(x) = c_3x^3 + c_2x^2 + c_1x^1 + c_0 = 3x^3 + 1x^2 + 1x + 2$. Такую операцию можно записать в матричном виде как

\begin{equation*}
	\left(
	\begin{array}{cccc}
		c_0 & c_3 & c_2 & c_1\\
		c_1 & c_0 & c_3 & c_2\\
		c_2 & c_1 & c_0 & c_3\\
		c_3 & c_2 & c_1 & c_0\\
	\end{array}
	\right) 
	\cdot
	\left(
	\begin{array}{c}
		a_0\\
		a_1\\
		a_2\\
		a_3\\
	\end{array}
	\right)
	=
	\left(
	\begin{array}{cccc}
		2 & 3 & 1 & 1\\
		1 & 2 & 3 & 1\\
		1 & 1 & 2 & 3\\
		3 & 1 & 1 & 2\\
	\end{array}
	\right) 
	\cdot
	\left(
	\begin{array}{c}
		a_0\\
		a_1\\
		a_2\\
		a_3\\
	\end{array}
	\right)
	=
	\left(
	\begin{array}{c}
		b_0\\
		b_1\\
		b_2\\
		b_3\\
	\end{array}
	\right).
\end{equation*}

Схема представлена на рисунке \ref{img:MX}.

\imgScale{0.8}{MX}{Схема MixColumns}.
\FloatBarrier

\textbf{AddRoundKey} применяет XOR к ключу раунда и внутреннему состоянию. Без KeyExpansion во всех раундах использовался бы один и тот ключ K.

Схема представлена на рисунке \ref{img:ARK}.

\imgScale{0.8}{ARK}{Схема AddRoundKey}.
\FloatBarrier

Для расшифрования шифротекста все используемые шифрующие преобразования могут быть инвертированы и применены в обратном порядке. Перед первым раундом дешифрования выполняется операция AddRoundKey , накладывающая на шифротекст четыре последних слова расширенного ключа. Порядок: InvShiftRows, InvSubBytes, AddRoundKey, InvMixColumns. В последнем раунде не выполняется InvMixColumns.

Раундовые ключи вырабатываются из ключа шифра K с помощью процедуры расширения ключа, в результате чего формируется массив раундовых ключей, из которого затем непосредственно выбирается необходимый раундовый ключ. Каждый раундовый ключ имеет длину 128 бит (или 4 32 битных слова слова $w_i$, $w_{i+1}$, $w_{i+2}$, $w_{i+3}$). Первые четыре слова являются ключом с номером 0. Новые слова $w_{i+4}$, $w_{i+5}$, $w_{i+6}$, $w_{i+7}$ следующего раундового ключа определяются из слов$w_i$, $w_{i+1}$, $w_{i+2}$, $w_{i+3}$ предыдущего ключа на основе уравнения $w_{i+j} = w_{i+j-1} \oplus w_{i+j-4}, j=5,7$. Первое слово $w_{i+4}$ в каждом ключе получается как $w_{i+4} = w_{i} \oplus g(w_{i+3})$. Действие функции g сводится к последовательному
выполнению трёх шагов, отображающих слово в слово:
\begin{itemize}[label=---]
	\item циклический сдвиг четырехбайтового слова влево на один байт;
	\item замена каждого байта слова соответствии с таблицей из SubBytes;
	\item суммирование по модулю 2 байтов, раундовой постоянной заданной таблицей.
\end{itemize}
Цель суммирования с раундовыми константами --- разрушить любую симметрию, что может возникнуть на разных этапах разворачивания ключа. Схема представлена на рисунке \ref{img:KE}.

\imgScale{0.63}{KE}{Схема алгоритма расширения ключа}.
\FloatBarrier

\section{Режим шифрования CFB}

Режим обратной связи по шифротексту или режим гаммирования с обратной связью (CFB) --- один из вариантов использования симметричного блочного шифра, при котором для шифрования следующего блока открытого текста он складывается по модулю 2 с перешифрованным результатом шифрования предыдущего блока. Ошибка, которая возникает в шифротексте при передаче, сделает невозможным расшифровку как блока, в котором ошибка произошла, так и следующего за ним, однако не распространяется на последующие блоки. Криптостойкость определяется криптостойкостью используемого шифра. Возможности распараллеливания процедуры шифрования ограничены, можно распараллелить расшифровку. Схемы шифрования и расшифровки представлены на рисунках \ref{img:enc} и \ref{img:dec}.

\imgScale{0.6}{enc}{Шифрование с CFB}
\imgScale{0.6}{dec}{Дешифрование с CFB}
\FloatBarrier

\section*{Вывод}

В данном разделе был рассмотрен алгоритм шифрования AES с использованием режима шифрования CFB.

