\chapter*{Введение}
\addcontentsline{toc}{chapter}{Введение}

Шифрование информации --- занятие, которым человек занимался ещё до начала первого тысячелетия, занятие, позволяющее защитить информацию от посторонних лиц. 

Существует множество методов шифрования. Одним из них является блочное шифрование. Блочные шифры оперируют группами бит фиксированной длины --- блоками, характерный размер которых меняется в пределах 64--256 бит. Одним из самых используемых был шифр DES. Это алгоритм для симметричного шифрования, разработанный фирмой IBM. Размер блока для DES равен 64 битам. В основе алгоритма лежит сеть Фейстеля с 16 раундами и ключом, имеющим длину 56 бит. Алгоритм использует комбинацию нелинейных (S-блоки) и линейных (перестановки E, IP, IP-1) преобразований.

\textbf{Цель} --- реализация программы шифрования симметричным алгоритмом DES с применением режима шифрования PCBC.

Для достижения поставленной цели необходимо выполнить следующие задачи:
\begin{itemize}[label=---]
	\item изучить алгоритм работы алгоритма DES;
	\item изучить режим шифрования PCBC;
	\item реализовать в виде программы алогритм шифрования DES с применением режима шифрования PCBC;
	\item обеспечить шифрование и расшифровку произвольного файла с использованием разработанной программы;
	\item предусмотреть работу программы с пустым, однобайтовым файлом и с файлами архива (rar, zip или др.).
\end{itemize}

