\chapter{Аналитическая часть}

\section{Алгоритм DES}

Data Encryption Standard (DES) --- это стандарт шифрования данных, изобретенный в США в 80-х годах ХХ века. DES перестал быть пригодным в условиях сверхбыстрой техники и больших объемов данных из-за ограничений в 56 бит на ключ и 64 бит на данные. Однако он все еще используется.

Хотя DES признан устаревшим и не удовлетворяющим современным требованиям, он может быть использован, например, в виде 3DES, когда шифр применяется три раза подряд. Такой подход снимает ограничение в размере ключа, но блок шифруемых данных остается прежним. В свое время DES был достаточно быстрым и криптоустойчивым шифром. Сейчас это не так, а 3DES и вовсе работает втрое медленнее. Несмотря на это DES по-прежнему используется в ряде систем, но его применение в новых проектах запрещено. 

Перед началом шифрования исходный текст разбивается на блоки по 64 бита. Если размер текста не кратен 64, то последний блок дополняется. Далее каждый блок проходит начальную перестановку по специальной таблице. Далее каждый блок проходит 16 раундом по схеме Фейстеля. После всех циклов осуществляется конечная перестановка и получается шифротекст. 

Каждый раунд блок разбивается пополам. Правый блок и ключ раунда проходят через основную функция шифрования и делают XOR с левым блоком. Перед началом следующего раунда половинки меняются местами.
 
В самой функции правый блок проходит через расширяющую функцию до 48 бит и делается XOR c ключом раунда. Результат разбивается на 8 подблоков по 6 бит. Каждый подблок преобразовывается с помощью своей таблицы подстановки в новый 4 битовый блок. Для подстановки из подблока выделяются 2 крайних и 4 серединных бита. Крайние биты склеиваются и получается номер строки в соответствующем S-блоке, а серединные номером столбца. Далее осуществляется перестановка. Схема работы функции представлена на рисунке \ref{img:F}.

\imgScale{0.55}{F}{Схема работы функции f}
\FloatBarrier

Для работы DES необходимо расширить 1 64 битный ключ до 16 48 битных ключей. Сначала ключ проходит сжимающую перестановку до 56 бит, выкидывается каждый 8 бит. Ключ шифра DES имеет длину 64 бита, но каждый восьмой предназначается лишь для контроля чётности. Далее он разбивается пополам. В каждом раунде половинки сначала циклически сдвигаются, затем объединятся и проходят сжимающую перестановку. Схема работы алгоритма генерации ключей представлена на рисунке \ref{img:key}.

\imgScale{0.55}{key}{Схема алгоритма генерации ключей}
\FloatBarrier

Общая схема шифрования алгоритма DES представлена на рисунке \ref{img:DES}.

\imgScale{0.6}{DES}{Схема шифрования}
\FloatBarrier

Основной недостаток DES – короткий ключ. Для решения этой проблемы был создан алгоритм 3DES с ключом длинной 192 бита. По сути является трёхкратным применением DES.

\section{Режим шифрования PCBC}

Режим распространяющегося сцепления блоков шифра. Этот режим похож на CBC за исключением того, что предыдущий блок открытого текста и предыдущий блок шифротекста подвергается операции XOR с текущим блоком открытого текста перед шифрованием или после него. Был разработан для того, чтобы небольшие изменения в зашифрованном тексте распространялись бесконечно как при расшифровке, так и при шифровании. Схема применения режима PCBC представлена на рисунках \ref{img:enc} и \ref{img:dec}.

\imgScale{0.8}{enc}{Шифрование с PCBC}
\imgScale{0.8}{dec}{Дешифрование с PCBC}
\FloatBarrier

\section*{Вывод}

В данном разделе был рассмотрен алгоритм шифрования DES с использованием режима шифрования PCBC.

