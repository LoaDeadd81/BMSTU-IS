\chapter{Технологическая часть}

В данном разделе будут рассмотрены средства реализации, а также представлены листинги реализаций алгоритма шифрования DES.

\section{Средства реализации}
В данной работе для реализации был выбран язык программирования $C$. Данный язык удоволетворяет поставленным критериям по средствам реализации.

\section{Реализация алгоритма}

В листингах \ref{lst:encode} и \ref{lst:decode} представлена реализация алгоритма шифрования DES с использованием режима PCBC.

\begin{center}
    \captionsetup{justification=raggedright,singlelinecheck=off}
    \begin{lstlisting}[label=lst:encode,caption=Шифрование DES]
size_t encode(uint8_t *to, uint8_t *keys8b, uint8_t *from, size_t length) {
	length = length % 8 == 0 ? length : length + (8 - (length % 8));
	
	uint64_t keys48b[16] = {0};
	uint32_t N1, N2;
	
	key_expansion(join_8bits_to_64bits(keys8b),keys48b);
	
	uint64_t lastCypher = 1, lastOpen = 1;
	for (size_t i = 0; i < length; i += 8) {
		uint64_t open = join_8bits_to_64bits(from + i);
		
		uint64_t tmp = lastOpen;
		lastOpen = open;
		
		open ^= lastCypher;
		open ^= tmp;
		
		split_64bits_to_32bits(initial_permutation(open), &N1, &N2);
		feistel_encode(&N1, &N2, keys48b);
		
		lastCypher = final_permutation(join_32bits_to_64bits(N1, N2));
		split_64bits_to_8bits(lastCypher, (to + i));
	}
	
	return length;
}
\end{lstlisting}
\end{center}

\begin{center}
\captionsetup{justification=raggedright,singlelinecheck=off}
\begin{lstlisting}[label=lst:decode,caption=Дешифрование DES]
size_t decode(uint8_t *to, uint8_t *keys8b, uint8_t *from, size_t length) {
	length = length % 8 == 0 ? length : length + (8 - (length % 8));
	
	uint64_t keys48b[16] = {0};
	uint32_t N1, N2;
	
	key_expansion(join_8bits_to_64bits(keys8b),keys48b);
	
	uint64_t lastCypher = 1, lastOpen = 1;
	for (size_t i = 0; i < length; i += 8) {
		uint64_t cypher = join_8bits_to_64bits(from + i);
		
		split_64bits_to_32bits(initial_permutation(cypher),&N1, &N2);
		feistel_decode(&N1, &N2, keys48b);
		
		uint64_t open = final_permutation(join_32bits_to_64bits(N1, N2));
		open ^= lastOpen;
		open ^= lastCypher;
		
		lastOpen = open;
		lastCypher = cypher;
		
		split_64bits_to_8bits(lastOpen,(to + i));
	}
	
	return length;
}
\end{lstlisting}
\end{center}
\FloatBarrier

\section{Тестирование}

Для тестирование написанной программы файл шифровался и дешифровался и сравнивалось их содержимое.

\begin{table}[h]
	\begin{center}
		\begin{threeparttable}
			\captionsetup{justification=raggedright,singlelinecheck=off}
			\caption{\label{tbl:functional_} Функциональные тесты}
			\begin{tabular}{|c|c|c|c|}
				\hline
				Входной файл & Ожидаемый результат & Результат \\ 
				\hline
				1\_in.txt (9 bytes) & 1\_out.txt (9 bytes) & 1\_out.txt (9 bytes)\\
				<<encode me>> & <<encode me>> &<<encode me>>\\
				\hline
				2\_in.txt (1 bytes) & 2\_out.txt (1 bytes) & 2\_out.txt (1 bytes)\\
				<<a>> & <<a>> &<<a>>\\
				\hline
				3\_in.txt (0 bytes) & 3\_out.txt (0 bytes) & 3\_out.txt (0 bytes)\\
				<<>> & <<>> &<<>>\\
				\hline
				4\_in.tar.gz (21 Kb) & 4\_out.tar.gz (21 Kb) & 4\_out.tar.gz (21 Kb)\\
				test.png & test.png & test.png\\
				\hline
			\end{tabular}
		\end{threeparttable}
	\end{center}
\end{table}
\FloatBarrier

\section*{Вывод}

Были представлен листинг реализации алгоритма работы энигмы. Также в данном разделе была приведена информация о выбранных средствах для разработки алгоритмов.
