\chapter{Аналитическая часть}
В этом разделе будет рассмотрено устройство шифровальной машины <<Энигма>> и ей комплектующих и приведён пример её работы.


\section{Основные детали}

Шифровальная машина <<Энигма>> состоит из следующих деталей: роторы, рефлектор, а также коммутационная панель.

\subsection{Роторы}

Ротор --- диск с 26 зубьями, для регулировки, и 26 контактами с обоих сторон (26 как букв в алфавите, количество может быть любым). Ротор производил подстановку входного символа. В результате прохождение через ротор символ менялся на другой и поступал дале по цепочке в следующий ротор или рефлектор. В конфигурации, используемой во Второй мировой, три ротора с 26 зубьями крепились на шпиндель и вставлялись в машинку. Схема работы роторов изображена на рисунке \ref{img:rotors}.


\imgScale{0.44}{rotors}{Схема работы роторов с рефлектором}
\FloatBarrier


\subsection{Рефлектор}

Рефлектор --- элемент, попарно соединяющий контакты последнего ротора, тем самым направляя ток обратно на последний ротор. Так, после этого электрический сигнал пойдёт в обратном направлении, пройдя через все роторы повторно. 

\subsection{Коммутационная панель}

Коммутационная панель позволяет оператору шифровальной машины варьировать содержимое проводов, попарно соединяющих буквы английского алфавита. Эффект состоял в том, чтобы усложнить работу машины, не увеличивая число роторов. Так, если на коммутационной панели соединены буквы 'A' и 'Z', то каждая буква 'A', проходящая через коммутационную панель, будет заменена на 'Z'  и наоборот. Сигналы попадали на коммутационную панель 2 раза: в начале и в конце обработки отдельного символа.

\section{Алгоритм работы}

При нажатии клавиши электрический сигнал проходит через коммутационную панель, роторы, рефлектор и идёт в обратном направлении чтоб включить лампочку на панели. Подсвеченная буква и будет являться зашифрованным символом. После каждой обработанной буквы роторы сдвигаются следующим образом: самый первый ротор проворачивается на одно деление после каждого нажатия клавиши, второй после полного оборота первого и тд. Для дешифрации зашифрованного сообщения, необходимо применить алгоритм шифрования энигмы к зашифрованному сообщению с настройками, которые имели место при шифровании исходного сообщения. Пример шифрования сообщения <<AAA>> для алфавита <<ABCDEFGH>> показан на рисунке \ref{img:123}.

\imgScale{0.63}{123}{Пример шифрования}
\FloatBarrier

\section*{Вывод}

В данном разделе были рассмотрены алгоритм работы шифровальной машины <<Энигма>>, использовавшейся во время Второй мировой войны.


