\chapter{Технологическая часть}

В данном разделе будут рассмотрены средства реализации, а также представлены листинги реализаций алгоритма шифрования машины <<Энигма>>.

\section{Средства реализации}
В данной работе для реализации был выбран язык программирования $C$. Данный язык удоволетворяет поставленным критериям по средствам реализации.

\section{Реализация алгоритма}

В листингах \ref{lst:enigma1} представлена реализация алгоритма шифрования машины <<Энигма>>.

\begin{center}
    \captionsetup{justification=raggedright,singlelinecheck=off}
    \begin{lstlisting}[label=lst:enigma1,caption=Реализация алгоритма шифрования машины <<Энигма>>]
typedef struct enigma_t {
	unsigned int rotor_size;
	unsigned int rotor_num;
	unsigned int *commutator;
	unsigned int *reflector;
	unsigned int **rotors;
	unsigned int *indexes;
} enigma_t;

enigma_t *new_enigma(unsigned int rotor_num, unsigned int rotor_size) {
	enigma_t *enigma = malloc(sizeof(enigma_t));
	if (enigma == NULL) return NULL;
	
	
	enigma->commutator = malloc(sizeof(unsigned int) * rotor_size);
	if (enigma->commutator == NULL) {
		free(enigma);
		return NULL;
	}
	enigma->reflector = malloc(sizeof(unsigned int) * rotor_size);
	if (enigma->reflector == NULL) {
		free(enigma->commutator);
		free(enigma);
		return NULL;
	}
	
	enigma->rotors = malloc(sizeof(unsigned int *) * rotor_num);
	for (int i = 0; i < rotor_num; ++i) enigma->rotors[i] = malloc(sizeof(unsigned int) * rotor_size);
	
	enigma->indexes = calloc(rotor_num, sizeof(unsigned int));
	
	enigma->rotor_size = rotor_size;
	enigma->rotor_num = rotor_num;
	
	return enigma;
}

void free_enigma(enigma_t *enigma) {
	for (int i = 0; i < enigma->rotor_num; ++i) free(enigma->rotors[i]);
	free(enigma->rotors);
	free(enigma->reflector);
	free(enigma->commutator);
	free(enigma->indexes);
	free(enigma);
}

void set_commutator(enigma_t *enigma, const unsigned int *commutator) {
	for (int i = 0; i < enigma->rotor_size; ++i) enigma->commutator[i] = commutator[i];
}

void set_rotors(enigma_t *enigma, unsigned int **rotors) {
	for (int i = 0; i < enigma->rotor_num; ++i)
	for (int j = 0; j < enigma->rotor_size; ++j)
	enigma->rotors[i][j] = rotors[i][j];
	
}

void set_reflector(enigma_t *enigma, const unsigned int *reflector) {
	for (int i = 0; i < enigma->rotor_size; ++i) enigma->reflector[i] = reflector[i];
}

unsigned int encrypt(enigma_t *enigma, unsigned int symbol, int *status) {
	*status = 1;
	if (enigma == NULL) {
		*status = 0;
		return 0;
	}
	if (symbol >= enigma->rotor_size) {
		*status = 0;
		return 0;
	}
	
	symbol = enigma->commutator[symbol];
	for (int i = 0; i < enigma->rotor_num; ++i)
	symbol = enigma->rotors[i][(symbol + enigma->indexes[i]) % enigma->rotor_size];
	
	symbol = enigma->reflector[symbol];
	
	for (int i = enigma->rotor_num - 1; i >= 0; --i) {
		symbol = backtrack(enigma, symbol, i, status);
		if (*status == 0) return 0;
	}
	symbol = enigma->commutator[symbol];
	
	int shift = 1;
	for (int i = 0; i < enigma->rotor_num && shift == 1; ++i) {
		shift = 0;
		if (enigma->indexes[i] >= enigma->rotor_size - 1) shift = 1;
		enigma->indexes[i] = (enigma->indexes[i] + 1) % enigma->rotor_size;
	}
	
	return symbol;
}

unsigned int backtrack(enigma_t *enigma, unsigned int symbol, unsigned int rotor_index, int *status) {
	*status = 1;
	
	for (int i = 0; i < enigma->rotor_size; ++i)
	if (enigma->rotors[rotor_index][i] == symbol)
	return i < enigma->indexes[rotor_index] ? enigma->rotor_size - (enigma->indexes[rotor_index] - i) : i - enigma->indexes[rotor_index];
	
	*status = 0;
	return 0;
}

\end{lstlisting}
\end{center}
\FloatBarrier

\section{Тестирование}

Для тестирование написанной программы можно воспользоваться тем, что для декодирования информации нужно прогнать зашифрованный текст через машину ещё раз с теми же настройками, что использовались для кодирования. Поэтому для тестирования входной файл кодируется и результат записывается во временный файл, этот файл кодируется ещё раз с теми же настройка и записывается в выходной файл. В результате входной и выходной файлы должны совпадать.

\begin{table}[h]
	\begin{center}
		\begin{threeparttable}
			\captionsetup{justification=raggedright,singlelinecheck=off}
			\caption{\label{tbl:functional_} Функциональные тесты}
			\begin{tabular}{|c|c|c|c|}
				\hline
				Входной файл & Ожидаемый результат & Результат \\ 
				\hline
				1\_in.txt (9 bytes) & 1\_out.txt (9 bytes) & 1\_out.txt (9 bytes)\\
				<<encode me>> & <<encode me>> &<<encode me>>\\
				\hline
				2\_in.txt (1 bytes) & 2\_out.txt (1 bytes) & 2\_out.txt (1 bytes)\\
				<<a>> & <<a>> &<<a>>\\
				\hline
				3\_in.txt (0 bytes) & 3\_out.txt (0 bytes) & 3\_out.txt (0 bytes)\\
				<<>> & <<>> &<<>>\\
				\hline
				4\_in.tar.gz (150 bytes) & 4\_out.tar.gz (150 bytes) & 4\_out.tar.gz (150 bytes)\\
				tst.txt (32 bytes)& tst.txt (32 bytes &tst.txt (32 bytes\\
				<<encode me>> & <<encode me>> &<<encode me>>\\
				\hline
			\end{tabular}
		\end{threeparttable}
	\end{center}
\end{table}
\FloatBarrier

\section*{Вывод}

Были представлен листинг реализации алгоритма работы энигмы. Также в данном разделе была приведена информация о выбранных средствах для разработки алгоритмов.
