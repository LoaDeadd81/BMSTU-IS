\chapter{Конструкторская часть}
В этом разделе будут представлены описания используемых типов данных, а также схема алгоритма разрабатываемой программы.

\section{Описание используемых типов данных}

При реализации алгоритмов будут использованы следующие типы данных для соответствующих значений:
\begin{itemize}[label=---]
	\item набор роторов --- матрица;
	\item рефлектор --- массив;
    \item коммутатор --- массив;
	\item сообщение --- последовательность байт входного файла.
\end{itemize}

\section{Сведения о модулях программы}
Программа состоит из двух модулей:
\begin{enumerate}[label=\arabic*)]
    \item $main$ --- файл, содержащий точку входа;
    \item $enigma$ --- файл, содержащий реализацию <<энигмы>>.
\end{enumerate}

\section{Разработка алгоритмов}
На рисунке \ref{img:enigma} представлена схема работы программы, реализующей шифровальную машину <<Энигма>>.

\imgScale{0.8}{enigma}{Схема алгоритма шифрования, используемого в программе}
\FloatBarrier

Алгоритм шифрования реализован для  алфавита размером 256, что необходимо для работы с данными файлов любых форматов.

\section*{Вывод}

В данном разделе были представлены описания используемых типов данных, а также схема алгоритма разрабатываемой программы.

